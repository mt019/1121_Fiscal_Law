% Options for packages loaded elsewhere
\PassOptionsToPackage{unicode}{hyperref}
\PassOptionsToPackage{hyphens}{url}
%
\documentclass[
]{book}
\usepackage{amsmath,amssymb}
\usepackage{lmodern}
\usepackage{iftex}
\ifPDFTeX
  \usepackage[T1]{fontenc}
  \usepackage[utf8]{inputenc}
  \usepackage{textcomp} % provide euro and other symbols
\else % if luatex or xetex
  \usepackage{unicode-math}
  \defaultfontfeatures{Scale=MatchLowercase}
  \defaultfontfeatures[\rmfamily]{Ligatures=TeX,Scale=1}
\fi
% Use upquote if available, for straight quotes in verbatim environments
\IfFileExists{upquote.sty}{\usepackage{upquote}}{}
\IfFileExists{microtype.sty}{% use microtype if available
  \usepackage[]{microtype}
  \UseMicrotypeSet[protrusion]{basicmath} % disable protrusion for tt fonts
}{}
\makeatletter
\@ifundefined{KOMAClassName}{% if non-KOMA class
  \IfFileExists{parskip.sty}{%
    \usepackage{parskip}
  }{% else
    \setlength{\parindent}{0pt}
    \setlength{\parskip}{6pt plus 2pt minus 1pt}}
}{% if KOMA class
  \KOMAoptions{parskip=half}}
\makeatother
\usepackage{xcolor}
\usepackage{longtable,booktabs,array}
\usepackage{calc} % for calculating minipage widths
% Correct order of tables after \paragraph or \subparagraph
\usepackage{etoolbox}
\makeatletter
\patchcmd\longtable{\par}{\if@noskipsec\mbox{}\fi\par}{}{}
\makeatother
% Allow footnotes in longtable head/foot
\IfFileExists{footnotehyper.sty}{\usepackage{footnotehyper}}{\usepackage{footnote}}
\makesavenoteenv{longtable}
\usepackage{graphicx}
\makeatletter
\def\maxwidth{\ifdim\Gin@nat@width>\linewidth\linewidth\else\Gin@nat@width\fi}
\def\maxheight{\ifdim\Gin@nat@height>\textheight\textheight\else\Gin@nat@height\fi}
\makeatother
% Scale images if necessary, so that they will not overflow the page
% margins by default, and it is still possible to overwrite the defaults
% using explicit options in \includegraphics[width, height, ...]{}
\setkeys{Gin}{width=\maxwidth,height=\maxheight,keepaspectratio}
% Set default figure placement to htbp
\makeatletter
\def\fps@figure{htbp}
\makeatother
\setlength{\emergencystretch}{3em} % prevent overfull lines
\providecommand{\tightlist}{%
  \setlength{\itemsep}{0pt}\setlength{\parskip}{0pt}}
\setcounter{secnumdepth}{5}
\usepackage{booktabs}
\usepackage{amsthm}
\makeatletter
\def\thm@space@setup{%
  \thm@preskip=8pt plus 2pt minus 4pt
  \thm@postskip=\thm@preskip
}
\makeatother
\ifLuaTeX
  \usepackage{selnolig}  % disable illegal ligatures
\fi
\usepackage[]{natbib}
\bibliographystyle{plainnat}
\IfFileExists{bookmark.sty}{\usepackage{bookmark}}{\usepackage{hyperref}}
\IfFileExists{xurl.sty}{\usepackage{xurl}}{} % add URL line breaks if available
\urlstyle{same} % disable monospaced font for URLs
\hypersetup{
  pdftitle={1121 財政法專題研究},
  pdfauthor={林明鏘; 陳衍任},
  hidelinks,
  pdfcreator={LaTeX via pandoc}}

\title{1121 財政法專題研究}
\author{林明鏘 \and 陳衍任}
\date{2023-09-19}

\begin{document}
\maketitle

{
\setcounter{tocdepth}{1}
\tableofcontents
}
\hypertarget{preface}{%
\chapter*{Preface}\label{preface}}
\addcontentsline{toc}{chapter}{Preface}

財政法專題研究

112-1 學期《財政法專題研究》課堂筆記。采用R的bookdown製作,輸出格式為bookdown::gitbook。

\hypertarget{ux7b2cux4e00ux5468}{%
\chapter{W01\_0905}\label{ux7b2cux4e00ux5468}}

我報告:2023/10/17【特種基金】

財政法

有請各老師來演講

每個同學要報告

一萬字

今天就要決定修課人數,
不要輪到你報告又不來,老師會記恨你一輩子,推薦信不會寫給你

同學要講誠信原則

有些國家的法學院是沒有開財政法的

台灣的財政法非常糟糕

大法官解釋【預算】
391/520

財政法一定會很具體

要抽象的話,就是財政憲法,爲什麽公營事業要國營?爲什麽臺電缺電,臺電拆分?核電?

特種基金是政府的小金庫嗎?這樣的刻板印象對嗎?
政府采購支出

量大穩定的稅都被中央拿走了

中央地方財政收支劃分的不平衡

醇化的財產【柯】

國有財產,公營事業

國有財產法的使用收益處分

擎天崗的水牛是國有財產嗎

\begin{center}\rule{0.5\linewidth}{0.5pt}\end{center}

法學論文寫作格式

隨文注釋!

沒有注釋,就是自己創造發明的?學術倫理的問題

2023/10/17【特種基金】

參考文獻要完整,不要一直呆在家裏寫

判決不備理由,當然違背法令

論文不備理由,當然違背法令

不要寫愚生,你沒那麽笨,不要妄自菲薄,引喻失意,寫學生認爲就好了

報告要簡報,20分鐘内講完,要歸納重點。

法律最重要是歸納法,比較少演繹法

\begin{center}\rule{0.5\linewidth}{0.5pt}\end{center}

老師給我提議:各種環境的基金:土地(污染)基金,元大

\hypertarget{ux7b2cux4e8cux5468}{%
\chapter{W02\_0912}\label{ux7b2cux4e8cux5468}}

財政法,報告同學將報告在報告前一周周五上傳,每一位同學閲讀之後要提意見。

特種基金,可以專注於特定的類型的基金。有趣的基金:華航基金。官商勾結。進口鷄蛋的基金,農委會的附屬基金。寫一個小一點的題目。

農業發展基金可不可以去補助進口鷄蛋?符合他基金的建立目的?

每節課500字心得。

\hypertarget{ux8521ux8302ux5bc5-ux9810ux7b97ux6cd5}{%
\chapter{蔡茂寅 預算法}\label{ux8521ux8302ux5bc5-ux9810ux7b97ux6cd5}}

英國,爲了收入目的而創設的預算制度,爲了獲得收入,必須向收入來源的貴族講明,錢要花在那去

至少現在在我國,預算法是爲了支出,而非爲了收入

現在的稅收,已經不只是有預算的通過

稅收,已不單純依靠預算法的依據,不只是有預算的通過就有稅收的正當性,而有更嚴格的,個別的實定法的規範。

現在,收支兩方,都具有龐大的財政計劃的意涵

\begin{quote}
路上很多施工?(笑了,真的有
師:消化預算嘛
\end{quote}

預算的計劃性、法律性

預算的本體:科目+金額

\begin{itemize}
\tightlist
\item
  法律性

  \begin{itemize}
  \tightlist
  \item
    拘束力
  \item
    計劃趕不上變化:

    \begin{itemize}
    \tightlist
    \item
      預算流用
    \item
      追加預算

      \begin{itemize}
      \tightlist
      \item
        包括追加減
      \end{itemize}
    \end{itemize}
  \end{itemize}
\item
  計劃性

  \begin{itemize}
  \tightlist
  \item
    人家要看得懂

    \begin{itemize}
    \tightlist
    \item
      盡可能簡單、明確性
    \item
      人家都看得懂,才有拘束力
    \item
      要讓誰看懂?議會看得懂、一般人要看得懂
    \item
      市長看不懂,就充實議會的能力
    \item
      補足民意代表在專業上的欠缺
    \end{itemize}
  \item
    科目+金額

    \begin{itemize}
    \tightlist
    \item
      明確
    \item
      全國一致
    \item
      長期性的一致
    \end{itemize}
  \item
    事前性
  \end{itemize}
\end{itemize}

由誰編制?

\begin{itemize}
\tightlist
\item
  誰收錢誰花錢就誰編制
\item
  行政機關

  \begin{itemize}
  \tightlist
  \item
    專業性
  \item
    編制時就適度反映民意(理想)

    \begin{itemize}
    \tightlist
    \item
      實現政黨的政治目標
    \item
      持續性地爭取選民支持
    \item
      遺珠之憾
    \item
      民意容易操縱
    \end{itemize}
  \end{itemize}
\end{itemize}

希望在編制的時候就反映了民意,但只是理想,很難落實

送到議會去審議,這時接受民意代表的檢驗?

\begin{itemize}
\tightlist
\item
  議會(民意代表)做預算審議,科目調整權?沒有

  \begin{itemize}
  \tightlist
  \item
    金額,只需維持、刪減,不得增加!
  \item
    爲什麽?(看老師的書)
  \item
    行政機關(依據其性質,一定會盡量多編預算)
  \item
    議會是看門狗的角色
  \item
    不可能給他加多一點
  \end{itemize}
\item
  區別【主決議】、【附帶決議】?

  \begin{itemize}
  \tightlist
  \item
    可以回避主決議?
  \item
    形式的濫用
  \end{itemize}
\item
\end{itemize}

\begin{quote}
在南大,原本要研究預算,不給研究,於是看書、研究寶石、易經研究突飛猛進。
\end{quote}

\end{document}
