\documentclass[]{book}
\usepackage{lmodern}
\usepackage{amssymb,amsmath}
\usepackage{ifxetex,ifluatex}
\usepackage{fixltx2e} % provides \textsubscript
\ifnum 0\ifxetex 1\fi\ifluatex 1\fi=0 % if pdftex
  \usepackage[T1]{fontenc}
  \usepackage[utf8]{inputenc}
\else % if luatex or xelatex
  \ifxetex
    \usepackage{xltxtra,xunicode}
  \else
    \usepackage{fontspec}
  \fi
  \defaultfontfeatures{Ligatures=TeX,Scale=MatchLowercase}
\fi
% use upquote if available, for straight quotes in verbatim environments
\IfFileExists{upquote.sty}{\usepackage{upquote}}{}
% use microtype if available
\IfFileExists{microtype.sty}{%
\usepackage{microtype}
\UseMicrotypeSet[protrusion]{basicmath} % disable protrusion for tt fonts
}{}
\usepackage[unicode=true]{hyperref}
\hypersetup{
            pdftitle={1121 財政法專題研究},
            pdfauthor={林明鏘; 陳衍任},
            pdfborder={0 0 0},
            breaklinks=true}
\urlstyle{same}  % don't use monospace font for urls
\usepackage{natbib}
\bibliographystyle{plainnat}
\usepackage{longtable,booktabs}
% Fix footnotes in tables (requires footnote package)
\IfFileExists{footnote.sty}{\usepackage{footnote}\makesavenoteenv{long table}}{}
\IfFileExists{parskip.sty}{%
\usepackage{parskip}
}{% else
\setlength{\parindent}{0pt}
\setlength{\parskip}{6pt plus 2pt minus 1pt}
}
\setlength{\emergencystretch}{3em}  % prevent overfull lines
\providecommand{\tightlist}{%
  \setlength{\itemsep}{0pt}\setlength{\parskip}{0pt}}
\setcounter{secnumdepth}{5}
% Redefines (sub)paragraphs to behave more like sections
\ifx\paragraph\undefined\else
\let\oldparagraph\paragraph
\renewcommand{\paragraph}[1]{\oldparagraph{#1}\mbox{}}
\fi
\ifx\subparagraph\undefined\else
\let\oldsubparagraph\subparagraph
\renewcommand{\subparagraph}[1]{\oldsubparagraph{#1}\mbox{}}
\fi

% set default figure placement to htbp
\makeatletter
\def\fps@figure{htbp}
\makeatother

\usepackage{booktabs}
\usepackage{longtable}

\usepackage{framed,color}
\definecolor{shadecolor}{RGB}{248,248,248}

\renewcommand{\textfraction}{0.05}
\renewcommand{\topfraction}{0.8}
\renewcommand{\bottomfraction}{0.8}
\renewcommand{\floatpagefraction}{0.75}

\renewcommand\contentsname{目錄}
\renewcommand\listfigurename{圖目錄}
\renewcommand\listtablename{表目錄}

% \renewcommand{\cftsecfont}{\stxs} %设置section条目的字体

\let\oldhref\href
\renewcommand{\href}[2]{#2\footnote{\url{#1}}}

\makeatletter
\newenvironment{kframe}{%
\medskip{}
\setlength{\fboxsep}{.8em}
 \def\at@end@of@kframe{}%
 \ifinner\ifhmode%
  \def\at@end@of@kframe{\end{minipage}}%
  \begin{minipage}{\columnwidth}%
 \fi\fi%
 \def\FrameCommand##1{\hskip\@totalleftmargin \hskip-\fboxsep
 \colorbox{shadecolor}{##1}\hskip-\fboxsep
     % There is no \\@totalrightmargin, so:
     \hskip-\linewidth \hskip-\@totalleftmargin \hskip\columnwidth}%
 \MakeFramed {\advance\hsize-\width
   \@totalleftmargin\z@ \linewidth\hsize
   \@setminipage}}%
 {\par\unskip\endMakeFramed%
 \at@end@of@kframe}
\makeatother

\makeatletter
\@ifundefined{Shaded}{
}{\renewenvironment{Shaded}{\begin{kframe}}{\end{kframe}}}
\@ifpackageloaded{fancyvrb}{%
  % https://github.com/CTeX-org/ctex-kit/issues/331
  \RecustomVerbatimEnvironment{Highlighting}{Verbatim}{commandchars=\\\{\},formatcom=\xeCJKVerbAddon}%
}{}
\makeatother

\usepackage{makeidx}
\makeindex

\urlstyle{tt}

\usepackage{amsthm}
\makeatletter
\def\thm@space@setup{%
  \thm@preskip=8pt plus 2pt minus 4pt
  \thm@postskip=\thm@preskip
}
\makeatother

\frontmatter

\title{1121 財政法專題研究}
\author{林明鏘 \and 陳衍任}
\date{2023-09-19}

\begin{document}
\maketitle


\thispagestyle{empty}

\begin{center}

    
\end{center}

\setlength{\abovedisplayskip}{-5pt}
\setlength{\abovedisplayshortskip}{-5pt}

{
\setcounter{tocdepth}{2}
\tableofcontents
}
\hypertarget{preface}{%
\chapter*{Preface}\label{preface}}


財政法專題研究

112-1 學期《財政法專題研究》課堂筆記。采用R的bookdown製作,輸出格式為bookdown::gitbook。

\hypertarget{ux7b2cux4e00ux5468}{%
\chapter{W01\_0905}\label{ux7b2cux4e00ux5468}}

我報告:2023/10/17【特種基金】

財政法

有請各老師來演講

每個同學要報告

一萬字

今天就要決定修課人數,
不要輪到你報告又不來,老師會記恨你一輩子,推薦信不會寫給你

同學要講誠信原則

有些國家的法學院是沒有開財政法的

台灣的財政法非常糟糕

大法官解釋【預算】
391/520

財政法一定會很具體

要抽象的話,就是財政憲法,爲什麽公營事業要國營?爲什麽臺電缺電,臺電拆分?核電?

特種基金是政府的小金庫嗎?這樣的刻板印象對嗎?
政府采購支出

量大穩定的稅都被中央拿走了

中央地方財政收支劃分的不平衡

醇化的財產【柯】

國有財產,公營事業

國有財產法的使用收益處分

擎天崗的水牛是國有財產嗎

\begin{center}\rule{0.5\linewidth}{0.5pt}\end{center}

法學論文寫作格式

隨文注釋!

沒有注釋,就是自己創造發明的?學術倫理的問題

2023/10/17【特種基金】

參考文獻要完整,不要一直呆在家裏寫

判決不備理由,當然違背法令

論文不備理由,當然違背法令

不要寫愚生,你沒那麽笨,不要妄自菲薄,引喻失意,寫學生認爲就好了

報告要簡報,20分鐘内講完,要歸納重點。

法律最重要是歸納法,比較少演繹法

\begin{center}\rule{0.5\linewidth}{0.5pt}\end{center}

老師給我提議:各種環境的基金:土地(污染)基金,元大

\hypertarget{ux7b2cux4e8cux5468}{%
\chapter{W02\_0912}\label{ux7b2cux4e8cux5468}}

財政法,報告同學將報告在報告前一周周五上傳,每一位同學閲讀之後要提意見。

特種基金,可以專注於特定的類型的基金。有趣的基金:華航基金。官商勾結。進口鷄蛋的基金,農委會的附屬基金。寫一個小一點的題目。

農業發展基金可不可以去補助進口鷄蛋?符合他基金的建立目的?

每節課500字心得。

\hypertarget{ux8521ux8302ux5bc5-ux9810ux7b97ux6cd5}{%
\chapter{蔡茂寅 預算法}\label{ux8521ux8302ux5bc5-ux9810ux7b97ux6cd5}}

英國,爲了收入目的而創設的預算制度,爲了獲得收入,必須向收入來源的貴族講明,錢要花在那去

至少現在在我國,預算法是爲了支出,而非爲了收入

現在的稅收,已經不只是有預算的通過

稅收,已不單純依靠預算法的依據,不只是有預算的通過就有稅收的正當性,而有更嚴格的,個別的實定法的規範。

現在,收支兩方,都具有龐大的財政計劃的意涵

\begin{quote}
路上很多施工?(笑了,真的有
師:消化預算嘛
\end{quote}

預算的計劃性、法律性

預算的本體:科目+金額

\begin{itemize}
\tightlist
\item
  法律性

  \begin{itemize}
  \tightlist
  \item
    拘束力
  \item
    計劃趕不上變化:

    \begin{itemize}
    \tightlist
    \item
      預算流用
    \item
      追加預算

      \begin{itemize}
      \tightlist
      \item
        包括追加減
      \end{itemize}
    \end{itemize}
  \end{itemize}
\item
  計劃性

  \begin{itemize}
  \tightlist
  \item
    人家要看得懂

    \begin{itemize}
    \tightlist
    \item
      盡可能簡單、明確性
    \item
      人家都看得懂,才有拘束力
    \item
      要讓誰看懂?議會看得懂、一般人要看得懂
    \item
      市長看不懂,就充實議會的能力
    \item
      補足民意代表在專業上的欠缺
    \end{itemize}
  \item
    科目+金額

    \begin{itemize}
    \tightlist
    \item
      明確
    \item
      全國一致
    \item
      長期性的一致
    \end{itemize}
  \item
    事前性
  \end{itemize}
\end{itemize}

由誰編制?

\begin{itemize}
\tightlist
\item
  誰收錢誰花錢就誰編制
\item
  行政機關

  \begin{itemize}
  \tightlist
  \item
    專業性
  \item
    編制時就適度反映民意(理想)

    \begin{itemize}
    \tightlist
    \item
      實現政黨的政治目標
    \item
      持續性地爭取選民支持
    \item
      遺珠之憾
    \item
      民意容易操縱
    \end{itemize}
  \end{itemize}
\end{itemize}

希望在編制的時候就反映了民意,但只是理想,很難落實

送到議會去審議,這時接受民意代表的檢驗?

\begin{itemize}
\tightlist
\item
  議會(民意代表)做預算審議,科目調整權?沒有

  \begin{itemize}
  \tightlist
  \item
    金額,只需維持、刪減,不得增加!
  \item
    爲什麽?(看老師的書)
  \item
    行政機關(依據其性質,一定會盡量多編預算)
  \item
    議會是看門狗的角色
  \item
    不可能給他加多一點
  \end{itemize}
\item
  區別本預算、【主決議】、【附帶決議】?

  \begin{itemize}
  \tightlist
  \item
    可以回避主決議?
  \item
    形式的濫用
  \item
    附條件?

    \begin{itemize}
    \tightlist
    \item
      我都可以刪除了,爲什麽不可以附條件?
    \item
      不可以。附條件,應該是對於未來不確定之事件
    \item
      如果所附條件是聯係到一方之意志,這就不符合【附條件】之本質
    \end{itemize}
  \end{itemize}
\end{itemize}

\begin{quote}
在南大,原本要研究預算,不給研究,於是看書、研究寶石、易經研究突飛猛進。
\end{quote}

林:

預算計劃是最具有拘束力的行政計劃,要有彈性,要與時俱進

法律要有穩定性,要有明確性

\backmatter
\printindex

\end{document}
