
% \documentclass[UTF8,a4paper,14pt]{ctexart}
\usepackage[a4paper, margin={1in,1.5in}]{geometry}
\usepackage[fontsize=14pt]{fontsize}
\renewcommand{\footnotesize}{\fontsize{8pt}{11pt}\selectfont}
\usepackage{
  url}


\usepackage{xeCJK}
\usepackage{zhnumber}

\usepackage{titlesec}
\usepackage{titling}
\usepackage{fontspec}
\usepackage{newunicodechar}
\usepackage{tocloft} % adding the tocloft package for toc customization
\usepackage{enumitem}

\setcounter{tocdepth}{2} % table of content

\setcounter{secnumdepth}{5}

% note: I'm using different fonts only because I don't have yours
% \setCJKmainfont{Noto Serif CJK TC}
% \setCJKsansfont{Noto Sans CJK TC}
% \setCJKmonofont{Noto Mono CJK TC}


%中英文設定
%\usepackage{fontspec}
\setmainfont{TeX Gyre Termes}
\usepackage{xeCJK} %引用中文字的指令集
%\setCJKmainfont{PMingLiU}
\setCJKmainfont{DFKai-SB}
\setCJKmainfont[AutoFakeBold=4,AutoFakeSlant=.4]{DFKai-SB}   %設定軟體粗體及斜體
% \setmainfont{Times New Roman}
\setCJKmonofont{DFKai-SB}


\setlength{\parindent}{2em} %首行縮排兩個漢字距離
\usepackage{indentfirst}
% 預設第一段不首行縮排,如果想讓第一段首行縮排,則可以使用 \usepackage{indentfirst}。
% 如果想讓某一段不首行縮排,則可以在該段前加上 \noindent。
% 如果想讓整篇文章都首行不縮排,則:\setlength{\parindent}{0pt}




% \newcommand{\subsubsubsection}[1]{\paragraph{#1}\mbox{}\\}

% \newcommand\subsubsubsection{\@startsection{paragraph}{4}{\z@}{-2.5ex\@plus -1ex \@minus -.25ex}{1.25ex \@plus .25ex}{\normalfont\normalsize\bfseries}}

\titleclass{\subsubsubsection}{straight}[\subsubsection]
\newcounter{subsubsubsection}
% [subsubsection]
% \renewcommand\thesubsubsubsection{\thesubsubsection.\arabic{subsubsubsection}}
\renewcommand\thesubsubsubsection{\arabic{subsubsubsection}.}

\titleformat{\subsubsubsection}
  {\normalfont\normalsize\bfseries}{\thesubsubsubsection}{1em}{}

\titlespacing*{\subsubsubsection}
{0pt}{3.25ex plus 1ex minus .2ex}{1.5ex plus .2ex}


\makeatletter
% \def\l@subsubsubsection{\@dottedtocline{4}{7em}{4em}}
\def\toclevel@subsubsubsection{4}
\def\l@subsubsubsection{\@dottedtocline{4}{7em}{4em}}
\makeatother
% https://tex.stackexchange.com/questions/60209/how-to-add-an-extra-level-of-sections-with-headings-below-subsubsection

%終於達到我想要的效果了QQQQQ

\titleformat{\paragraph}
{\normalfont\normalsize\bfseries}{\theparagraph}{1em}{}
\titlespacing*{\paragraph}
{0pt}{3.25ex plus 1ex minus .2ex}{1.5ex plus .2ex}

% in your example the titles in the toc are all sans serif, so I'll just add that here
% feel free to leave that out in your original document,
% it's just for visual comparability
\renewcommand{\cftsecfont}{\bfseries\sffamily}
\renewcommand{\cftsubsecfont}{\sffamily}
\renewcommand{\cftsubsubsecfont}{\sffamily}
% \renewcommand{\cftsubsubsubsecfont}{\sffamily}
\renewcommand{\cftparafont}{\sffamily}
\renewcommand{\cftsubparafont}{\sffamily}

% zhnum[style={Traditional,Financial}] doesn't work with the section counter,
% so we define our own counter and increase it every time in \thesection
\newcounter{mysec}[section]
\renewcommand\thesection{%
    \addtocounter{mysec}{1}%
    \zhnum[style={Traditional,Financial}]{mysec}、}
\renewcommand\thesubsection{\zhnum{subsection}、} % added a 、
\renewcommand\thesubsubsection{(\zhnum{subsubsection})} % added parentheses
% \renewcommand\thesubsubsubsection{\zhnum{subsubsection}.} % added parentheses
% (full-width, don't know if that's what you want)
\renewcommand\theparagraph{} % you don't want paragraph numbers
\renewcommand\thesubparagraph{} % nor subparagraph numbers

% we have to adjust the spacing in the toc because the section label is longer than usual
\addtolength\cftsecnumwidth{1em}
\addtolength\cftsubsecindent{1em}
% \addtolength\cftsubsubsecindent{1em}

% here we need to make sure the normal section counter is accessed
\titleformat{\section}{\Large\bfseries\filcenter}
    {\zhnum[style={Traditional,Financial}]{section}、}{.5em}{}
% not really sure what you intend to achieve with \fontsize but I'll leave it here
\titleformat*{\subsection}{\fontsize{15}{20}\bfseries\sffamily} 
\titleformat*{\subsubsection}{\fontsize{14}{18}\bfseries\sffamily}
% \titleformat*{\subsubsubsection}{\fontsize{14}{18}\bfseries\sffamily}

% no extra version for numberless is necessary since no numbers are used anyways
% also you get newlines from omitting the [display] in \titleformat already
% \titleformat{\paragraph}
%     {\fontsize{14}{16}\bfseries\sffamily}{}{0em}{} 
% \titleformat{\subparagraph}
%     {\fontsize{12}{14}\bfseries\sffamily}{}{0em}{}
% we need the following so that they don't indent (second argument, 0em);
% you'll have to adjust the spacing though since this is not display style anymore:
% \titlespacing*{\paragraph}{0em}{3.25ex plus 1ex minus .2ex}{.75ex plus .1ex} 
% \titlespacing*{\subparagraph}{0em}{3.25ex plus 1ex minus .2ex}{.75ex plus .1ex}

% \renewcommand{\maketitlehooka}{\sffamily}

\renewcommand{\baselinestretch}{1.2}
\renewcommand{\abstractname}{摘要} 
\renewcommand{\contentsname}{\hfill\bfseries 目錄 \hfill} 
\renewcommand{\tablename}{表}
\renewcommand{\arraystretch}{1}


\usepackage{fancyhdr}%导入fancyhdr包

\usepackage{lastpage}
\pagestyle{fancy}
\fancyhf{} 
\cfoot{第 \thepage 頁,共\pageref*{LastPage} 頁}
\usepackage[hang,flushmargin,bottom]{footmisc} %
% \usepackage[]{footmisc}

% \usepackage{hyperref}
% \usepackage[utf8x]{inputenc} do not use inputenc with XeTeX
% \usepackage{fixltx2e} not required any more
\usepackage{graphicx}
\usepackage{longtable}
\usepackage{float}
\usepackage{wrapfig}
\usepackage{rotating}
\usepackage[normalem]{ulem}
\usepackage{amsmath}
\usepackage{textcomp}

\usepackage{multirow}
\usepackage{booktabs}

\usepackage{url}
\let\oldquote\quote
\let\endoldquote\endquote
\renewenvironment{quote}[2][]
  {\if\relax\detokenize{#1}\relax
     \def\quoteauthor{#2}%
   \else
     \def\quoteauthor{#2~---~#1}%
   \fi
   \oldquote}
  {\par\nobreak\smallskip\hfill(\quoteauthor)%
   \endoldquote\addvspace{\bigskipamount}}

   \usepackage{hyperref}
\hypersetup{
  colorlinks=true,
  linkcolor=[rgb]{0,0.37,0.53},
  citecolor=[rgb]{0,0.47,0.68},
  filecolor=[rgb]{0,0.37,0.53},
  urlcolor=[rgb]{0,0.37,0.53},
  % pagebackref=true, % this is ignored
  linktoc=all}

\usepackage{longtable}
\usepackage{array}
\usepackage{tabularray}
\newcolumntype{C}[1]{>{\centering\arraybackslash}p{#1}}


\usepackage{afterpage}
% \usepackage{ctex}
% \author{王逸帆}
% \date{\today}




