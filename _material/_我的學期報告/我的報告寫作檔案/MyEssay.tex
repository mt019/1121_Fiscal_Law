\documentclass[12pt,a4paper]{article}
% \documentclass[UTF8,a4paper,12pt]{ctexart}

% \documentclass[UTF8,a4paper,14pt]{ctexart}
\usepackage[a4paper, margin={1in,1.5in}]{geometry}
\usepackage[fontsize=12pt]{fontsize}
\renewcommand{\footnotesize}{\fontsize{10pt}{12pt}\selectfont}
\usepackage{
  url}

% \usepackage{ctex}
\usepackage{xeCJK}
\usepackage{zhnumber}

\usepackage{titlesec}
\usepackage{titling}
\usepackage{fontspec}
\usepackage{newunicodechar}
\usepackage{tocloft} % adding the tocloft package for toc customization
\usepackage{enumitem}

% \newcounter{mycounter}
\AddEnumerateCounter*{\chinese}{\zhnum}{1}
% \renewcommand\theenumi{\zhnum{enumi}}

\setcounter{tocdepth}{3} % table of content

\setcounter{secnumdepth}{5}

% note: I'm using different fonts only because I don't have yours
% \setCJKmainfont{Noto Serif CJK TC}
% \setCJKsansfont{Noto Sans CJK TC}
% \setCJKmonofont{Noto Mono CJK TC}


%中英文設定
%\usepackage{fontspec}
\setmainfont{TeX Gyre Termes}
\usepackage{xeCJK} %引用中文字的指令集
%\setCJKmainfont{PMingLiU}
\setCJKmainfont{DFKai-SB}
\setCJKmainfont[AutoFakeBold=4,AutoFakeSlant=.4]{DFKai-SB}   %設定軟體粗體及斜體
% \setmainfont{Times New Roman}
\setCJKmonofont{DFKai-SB}


\setlength{\parindent}{2em} %首行縮排兩個漢字距離
\usepackage{indentfirst}
% 預設第一段不首行縮排,如果想讓第一段首行縮排,則可以使用 \usepackage{indentfirst}。
% 如果想讓某一段不首行縮排,則可以在該段前加上 \noindent。
% 如果想讓整篇文章都首行不縮排,則:\setlength{\parindent}{0pt}


% \newcommand{\subsubsubsection}[1]{\paragraph{#1}\mbox{}\\}

% \newcommand\subsubsubsection{\@startsection{paragraph}{4}{\z@}{-2.5ex\@plus -1ex \@minus -.25ex}{1.25ex \@plus .25ex}{\normalfont\normalsize\bfseries}}

\titleclass{\subsubsubsection}{straight}[\subsubsection]
\newcounter{subsubsubsection}
% [subsubsection]
% \renewcommand\thesubsubsubsection{\thesubsubsection.\arabic{subsubsubsection}}
\renewcommand\thesubsubsubsection{\arabic{subsubsubsection}.}

\titleformat{\subsubsubsection}
  {\normalfont\normalsize\bfseries}{\thesubsubsubsection}{1em}{}

\titlespacing*{\subsubsubsection}
{0pt}{3.25ex plus 1ex minus .2ex}{1.5ex plus .2ex}


\makeatletter
% \def\l@subsubsubsection{\@dottedtocline{4}{7em}{4em}}
\def\toclevel@subsubsubsection{4}
\def\l@subsubsubsection{\@dottedtocline{4}{7em}{4em}}
\makeatother
% https://tex.stackexchange.com/questions/60209/how-to-add-an-extra-level-of-sections-with-headings-below-subsubsection

%終於達到我想要的效果了QQQQQ

\titleformat{\paragraph}
{\normalfont\normalsize\bfseries}{\theparagraph}{1em}{}
\titlespacing*{\paragraph}
{0pt}{3.25ex plus 1ex minus .2ex}{1.5ex plus .2ex}

% in your example the titles in the toc are all sans serif, so I'll just add that here
% feel free to leave that out in your original document,
% it's just for visual comparability
\renewcommand{\cftsecfont}{\bfseries\sffamily}
\renewcommand{\cftsubsecfont}{\sffamily}
\renewcommand{\cftsubsubsecfont}{\sffamily}
% \renewcommand{\cftsubsubsubsecfont}{\sffamily}
\renewcommand{\cftparafont}{\sffamily}
\renewcommand{\cftsubparafont}{\sffamily}

% zhnum[style={Traditional,Financial}] doesn't work with the section counter,
% so we define our own counter and increase it every time in \thesection
\newcounter{mysec}[section]
\renewcommand\thesection{%
    \addtocounter{mysec}{1}%
    \zhnum[style={Traditional,Financial}]{mysec}、}
\renewcommand\thesubsection{\zhnum{subsection}、} % added a 、
\renewcommand\thesubsubsection{(\zhnum{subsubsection})} % added parentheses
% \renewcommand\thesubsubsubsection{\zhnum{subsubsection}.} % added parentheses
% (full-width, don't know if that's what you want)
\renewcommand\theparagraph{} % you don't want paragraph numbers
\renewcommand\thesubparagraph{} % nor subparagraph numbers

% we have to adjust the spacing in the toc because the section label is longer than usual
\addtolength\cftsecnumwidth{1em}
\addtolength\cftsubsecindent{1em}
% \addtolength\cftsubsubsecindent{1em}

% here we need to make sure the normal section counter is accessed
\titleformat{\section}{\Large\bfseries\filcenter}
    {\zhnum[style={Traditional,Financial}]{section}、}{.5em}{}
% not really sure what you intend to achieve with \fontsize but I'll leave it here
\titleformat*{\subsection}{\fontsize{15}{20}\bfseries\sffamily} 
\titleformat*{\subsubsection}{\fontsize{14}{18}\bfseries\sffamily}
% \titleformat*{\subsubsubsection}{\fontsize{14}{18}\bfseries\sffamily}

% no extra version for numberless is necessary since no numbers are used anyways
% also you get newlines from omitting the [display] in \titleformat already
% \titleformat{\paragraph}
%     {\fontsize{14}{16}\bfseries\sffamily}{}{0em}{} 
% \titleformat{\subparagraph}
%     {\fontsize{12}{14}\bfseries\sffamily}{}{0em}{}
% we need the following so that they don't indent (second argument, 0em);
% you'll have to adjust the spacing though since this is not display style anymore:
% \titlespacing*{\paragraph}{0em}{3.25ex plus 1ex minus .2ex}{.75ex plus .1ex} 
% \titlespacing*{\subparagraph}{0em}{3.25ex plus 1ex minus .2ex}{.75ex plus .1ex}

% \renewcommand{\maketitlehooka}{\sffamily}

\renewcommand{\baselinestretch}{1.2}
\renewcommand{\abstractname}{摘要} 
\renewcommand{\contentsname}{\hfill\bfseries 目錄 \hfill} 
\renewcommand{\listtablename}{\hfill\bfseries 表目錄 \hfill}
\renewcommand{\listfigurename}{\hfill\bfseries 圖目錄 \hfill}
\renewcommand{\tablename}{表}
\renewcommand{\figurename}{圖}
\renewcommand{\arraystretch}{1}


\usepackage{fancyhdr}%导入fancyhdr包

\usepackage{lastpage}
\pagestyle{fancy}
\fancyhf{} 
\cfoot{第 \thepage 頁,共\pageref*{LastPage} 頁}
\usepackage[hang,flushmargin,bottom]{footmisc} %
% \usepackage[]{footmisc}

% \usepackage{hyperref}
% \usepackage[utf8x]{inputenc} do not use inputenc with XeTeX
% \usepackage{fixltx2e} not required any more
\usepackage{graphicx}
\usepackage{longtable}
\usepackage{float}
\usepackage{wrapfig}
\usepackage{rotating}
\usepackage[normalem]{ulem}
\usepackage{amsmath}
\usepackage{textcomp}

\usepackage{multirow}
\usepackage{booktabs}

\usepackage{url}
\let\oldquote\quote
\let\endoldquote\endquote
\renewenvironment{quote}[2][]
  {\if\relax\detokenize{#1}\relax
     \def\quoteauthor{#2}%
   \else
     \def\quoteauthor{#2~---~#1}%
   \fi
   \oldquote}
  {\par\nobreak\smallskip\hfill(\quoteauthor)%
   \endoldquote\addvspace{\bigskipamount}}

   \usepackage{hyperref}
\hypersetup{
  colorlinks=true,
  linkcolor=[rgb]{0,0.37,0.53},
  citecolor=[rgb]{0,0.47,0.68},
  filecolor=[rgb]{0,0.37,0.53},
  urlcolor=[rgb]{0,0.37,0.53},
  % pagebackref=true, % this is ignored
  linktoc=all}

\usepackage{longtable}
\usepackage{array}
\usepackage{tabularray}
\newcolumntype{C}[1]{>{\centering\arraybackslash}p{#1}}


\usepackage{afterpage}

% \author{王逸帆}
% \date{\today}



\usepackage[
  style=gb7714-2015ay-ntulaw, 
gbfootbibfmt = true,
sortlocale=zh__stroke,
citestyle=gb7714-2015ay-ntulaw,
 gbalign=gb7714-2015ay-ntulaw,
%  gbpunctcn =true,
 gbtype = false, mergedate=true, url = false, backend=biber, defernumbers = true]{biblatex}

 \AtEveryCitekey{\clearfield{pages}}

 \renewcommand{\bibfootnotewrapper}[1]{%
  \bibsentence #1}

% \usepackage[style=gb7714-2015ay, sortlocale=zh__stroke,citestyle=gb7714-2015ay, gbalign=right, gbtype = false, mergedate=true, url = false, backend=biber, defernumbers = true]{biblatex}
\addbibresource{R10A21126.bib}

\DeclareFieldFormat[book]{title}{\iffieldequalstr{userd}{chinese}{《#1》}{\mkbibquote{#1}\isdot}}
% \DeclareFieldFormat[online]{title}{#1}

\DeclareFieldFormat[article]{title}{〈#1〉\isdot}
\DeclareFieldFormat[thesis]{title}{〈#1〉\isdot}
\DeclareFieldFormat[online]{title}{〈#1〉\isdot}
% \DeclareFieldFormat[online]{institution}{《#1》\isdot}
% \DeclareFieldFormat[online]{urldate}{《#1》\isdot}
\DeclareFieldFormat{urldate}{最後瀏覽日:#1}
\DeclareFieldFormat{date}{#1}
% \DeclareFieldFormat{urldate}{\addcomma\space\bibstring{urlseen}\space#1}


% 调整参考文献条目的缩进
\setlength{\bibitemindent}{0em} % 调整首行缩进
\setlength{\bibhang}{0em} % 调整其余行的缩进


% %
% % 【著者-出版年制】文献表缩进控制
% \setlength{\bibitemindent}{-1em} % bibitemindent表示一条文献中第一行相对后面各行的缩进
% \setlength{\bibhang}{0pt} % 著者-出版年制中 bibhang 表示的各行起始位置到页边的距离
% %

%
% 【顺序编码制】文献表缩进控制
% 调整顺序标签与文献内容的间距
\setlength{\biblabelsep}{0mm}
\setlength{\bibitemindent}{0pt}
\setlength{\biblabelextend}{0pt}



%第一段代码:导言区重设标注标签的标点
%多个引用间的标点
\renewcommand*{\multicitedelim}{;}%\addsemicolon\addspace;
\renewcommand*{\compcitedelim}{,}%\addcomma\space
\renewcommand{\compextradelim}{,}
\renewcommand*{\volcitedelim}{:}
%姓名与年份之间的标点(间隔符)
\DeclareDelimFormat[cite,parencite,pagescite]{nameyeardelim}{,}
\DeclareDelimFormat[cite,parencite,pagescite]{urldateurldelim}{,}
\DeclareDelimFormat[textcite]{nameyeardelim}{,}%
%最后一个姓名与等之间的符号(间隔符)
\DeclareDelimFormat[cite,parencite,pagescite]{andothersdelim}{,}%
\DeclareDelimFormat[textcite]{andothersdelim}{,}%
\DeclareDelimFormat[cite,parencite,pagescite]{addperiod}{。}%
\DeclareDelimFormat{nametitledelim}{\addcomma\space}




%除此之外,有时还需要设置finalnamedelim等来调整姓名间的标点。
% multinamedelim是各姓名之间的标点
% finalnamedelim是最后一个姓名前的取代multinamedelim的标点


%第二段代码:导言区重设标注标签的标点,根据文献的中英文调整标点形式
\renewrobustcmd{\mkbibleftborder}
{\iffieldequalstr{userf}{chinese}{(}{(}}%
\renewrobustcmd{\mkbibrightborder}
{\iffieldequalstr{userf}{chinese}{)}{)}}%
\renewcommand*{\multicitedelim}{\iffieldequalstr{userf}{chinese}{;}{\addsemicolon\addspace}}%;
\renewcommand*{\compcitedelim}{\iffieldequalstr{userf}{chinese}{,}{\addcomma\space}}
\renewcommand{\compextradelim}{\iffieldequalstr{userf}{chinese}{,}{\addcomma\space}}
\DeclareDelimFormat[cite,parencite,pagescite,citep]{nameyeardelim}
{\iffieldequalstr{userf}{chinese}{,}{\addcomma\space}}%
\DeclareDelimFormat[textcite,authornumcite,citet]{nameyeardelim}
{\iffieldequalstr{userf}{chinese}{,}{\addcomma\space}}%
\DeclareDelimFormat[cite,parencite,pagescite,citep]{andothersdelim}
{\iffieldequalstr{userf}{chinese}{,}{\addcomma\space}}%
\DeclareDelimFormat[textcite,authornumcite,citet]{andothersdelim}
{\iffieldequalstr{userf}{chinese}{,}{\addcomma\space}}%




%%全局标点设置
\DeclareDelimFormat{nameyeardelim}{,}%\addcomma\addspace
\DeclareDelimFormat[bib,biblist]{nameyeardelim}{,}%\addcomma\addspace
\DeclareDelimFormat{bibpagespunct}{\iffieldequalstr{userd}{chinese}{,}{,}}%\addcomma\addspace
\renewcommand*{\newunitpunct}{,}%\addcomma\space %,
\renewcommand*{\finentrypunct}{\iffieldequalstr{userd}{chinese}{。}{。}}
% \renewcommand*{\finentrypunct}{\iffieldequalstr{userd}{chinese}{。}{\adddot}}
\renewcommand*{\bibpagerefpunct}{\iffieldequalstr{userd}{chinese}{。}{。}}
% \renewcommand*{\bibpagerefpunct}{\iffieldequalstr{userd}{chinese}{。}{\adddot}}
\DeclareDelimFormat{multinamedelim}{、}%[bib,biblist]
\DeclareDelimFormat{finalnamedelim}{,}
\DeclareDelimFormat{andothersdelim}{,}


% 标点类型的控制(注意:全局字体能控制标点的字体)
% 标题与文献类型表示符之间的间隔符 title<titletypedelim>[J]
\DeclareDelimFormat[bib,biblist]{titletypedelim}{\space}
%文献表各条文献中各单元间隔标点设置(与异步标点机制相关的)
% \renewcommand*{\newunitpunct}{,}%\addcomma\space %,
% % \renewcommand*{\finentrypunct}{。}
% \renewcommand*{\finentrypunct}{\iffieldequalstr{userd}{chinese}{。}{\adddot}}


%姓名格式相关的标点
\DeclareDelimFormat[bib,biblist]{nameyeardelim}{\addspace}%\addcomma\addspace
\DeclareDelimFormat[bib,biblist]{multinamedelim}{、}%[bib,biblist]
\DeclareDelimFormat[bib,biblist]{finalnamedelim}{、}
\DeclareDelimFormat[bib,biblist]{andothersdelim}{,}
%姓名内部的相关标点,包括如下等设置
%注意这类设置与gbnamefmt选项相关,不同的选项值对应不同的gb...localset
%除了如下命令外还有其它设置,有需要可以查biblatex文档
\def\gbcaselocalset{%
\renewcommand*{\revsdnamepunct}{,}%%
\renewrobustcmd*{\bibinitperiod}{}%将名字简写后的点去掉
\renewrobustcmd*{\bibinithyphendelim}{}%.\mbox{-}
\renewrobustcmd*{\bibnamedelima}{} %%\mbox{-}
}


%%本地化字符串设置
\def\str@qicn{期}
% \def\str@juancn{卷}


\NewBibliographyString{qicn}
\NewBibliographyString{juancn}
\DefineBibliographyStrings{english}{
        and         = {,},%将第2和3人名间的and符号改成逗号,
        andcn       = {,},%and本地化字符串的中文对应词
        qicn ={\str@qicn},
        juancn ={卷},
        in={},
}

%%中文文献的相关设置
\AtEveryBibitem{
% \ifboolexpr{%
% togl{bbx:gbstyle} or test {\iffieldequalstr{userd}{chinese}}%
% }%

% {%
%
%   调整期刊名的格式
%
%   调整期刊名的格式
%
%   v1.0k,20180425,增加了字体控制命令,hzz
\renewbibmacro*{journal+issuetitle}{\bibpubfont%源来自standard.bbx
  \usebibmacro{journal}%
  \setunit*{,}%修改为增加一个逗号
  \usebibmacro{issue+date}%
  \usebibmacro{volume+number+eid}%把卷期放到年份后面
  %\newunit
  }%
%
%   调整期刊卷和期的格式
%
% \renewbibmacro*{volume+number+eid}{%源来自standard.bbx
% \printfield{volume}%
% \bibstring{juancn}%
% \bibstring{serialcn}%
%   \printfield{number}%
%   \bibstring{qicn}%
% }%


\renewbibmacro*{volume+number+eid}{%
  \printfield{volume}%
  \iffieldundef{volume}
    {}
    {\bibstring{juancn}}% 只有当有卷号时才显示 "第N卷"
  \bibstring{serialcn}%
  \printfield{number}%
  \bibstring{qicn}%
}

% }{}
}%

\renewbibmacro*{publisher+location+date}{%
  \printlist{location}%
  \iflistundef{publisher}
    {\setunit*{,}}
    {\setunit*{,}}%
  \printlist{publisher}%
  % \setunit*{,}%
  % \usebibmacro{date}%
  \newunit}

  % \DeclareFieldFormat{pages}{#1}%页码引用格式的修改%去掉前面引导页码的pp.等字符

  \usepackage{zhlipsum}

  \usepackage{tabularx}

  \usepackage{ragged2e}

  \usepackage{tikz}
\usetikzlibrary{positioning, arrows.meta, shapes.geometric}


\usepackage{graphicx}
\usepackage{caption}
% \usepackage{xcolor}
% \usepackage{grffile}

% % Define a command to convert images to black and white
% \newcommand{\bwimage}[2][]{%
%   \includegraphics[#1]{#2}%
% }

\usepackage{siunitx} % For aligning numbers by decimal point
\usepackage{arydshln}
\author{王逸帆\,
% R10A21126
\thanks{國立臺灣大學法律學系研究所碩士班財稅法學組三年級,學號:R10A21126。}
\vspace{-60em}
}
\date{}
% \date{\ctexset{today=big}}
\title{特別公課與特種基金
% \\  \large —— 以空氣污染防制費收費辦法第17條為中心 
\thanks{
  112學年度第1學期「財政法專題研究」課堂報告,授課教師:林明鏘、陳衍任教授。}}


\setlength{\parskip}{1em}

\begin{document}

\maketitle
\makeatother

\vspace{1pt}

\begin{abstract}
\setlength{\parindent}{2em}
\noindent
\hspace*{0.9\parindent}

特種基金

檢討:財政紀律法8、因應財政紀律法設立非營業特種基金之執行原則

   \end{abstract}



\thispagestyle{empty} %封面頁不編頁碼
\clearpage
    

\tableofcontents 


\thispagestyle{empty} %封面頁不編頁碼
\clearpage
\setcounter{page}{1} %從正文開始編頁碼

\section{前言}
\subsection{研究背景}

為防制空氣污染,維護生活環境及國民健康,以提高生活品質,立法院於民國64年公布施行「空氣污染防制法」
% (下稱本法)
,歷經多次修法,對各空氣污染源徵收「空氣污染防制費」。行政院環保署依法律授權,訂定有「	空氣污染防制費收費辦法」
% (下稱收費辦法)
,亦經多次修正。
司法院大法官在86年釋字第 426 號解釋將「空氣污染防制費」定位為「特別公課」
\footnote{節錄該號解釋理由書:「憲法增修條文第九條(按:現移列第10條)第二項規定:「經濟及科學技術發展,應與環境及生態保護兼籌並顧」,係課國家以維護生活環境及自然生態之義務,防制空氣污染為上述義務中重要項目之一。空氣污染防制法之制定符合上開憲法意旨。依該法徵收之空氣污染防制費係本於污染者付費之原則,對具有造成空氣污染共同特性之污染源,徵收一定之費用,俾經由此種付費制度,達成行為制約之功能,減少空氣中污染之程度;並以徵收所得之金錢,在環保主管機關之下成立空氣污染防制基金,專供改善空氣品質、維護國民健康之用途。此項防制費既係國家為一定政策目標之需要,對於有特定關係之國民所課徵之公法上負擔,並限定其課徵所得之用途,在學理上稱為特別公課,乃現代工業先進國家常用之工具。
特別公課與稅捐不同,稅捐係以支應國家普通或特別施政支出為目的,以一般國民為對象,課稅構成要件須由法律明確規定,凡合乎要件者,一律由稅捐稽徵機關徵收,並以之歸入公庫,其支出則按通常預算程序辦理;特別公課之性質雖與稅捐有異,惟特別公課既係對義務人課予繳納金錢之負擔,故其徵收目的、對象、用途應由法律予以規定,其由法律授權命令訂定者,如授權符合具體明確之標準,亦為憲法之所許。」}。
解釋認為「空氣污染防制費」是一種特別公課,並
肯認特別公課係對於課徵義務人之公法上金錢負擔,作為一財政工具之類型,與稅捐公課為不同之財政工具,并且强調空氣污染防制費係本於污染者付費原則,有行為制約功能。

行政院環保署自85年度起,依當時空氣污染防制法施行細則第14條及預算法規定\footnote{現依空氣污染防制法第18條第2項規定、預算法第4條第1項第2款、第21條規定。},設置空氣污染防制基金\footnote{
  現行空氣污染防制基金收支保管及運用辦法(110 年 08 月 12 日)
  第 3 條:
本基金之來源如下:
一、依空氣污染防制費收費辦法由中央主管機關徵收之空氣污染防制費收入。
二、本署依本法第九條第一項第二款交易或拍賣所得。
三、本署依本法第八十六條追繳之所得利益及違反本法罰鍰之部分提撥。
四、依本法科處並繳納之罰金,及因違反本法規定沒收或追徵之現金或變賣所得。
五、本基金之孳息收入。
六、其他有關收入。},執行空氣污染之防制、改善、國際環保、空氣品質監測等相關工作。為因應業務量變動所需,於88年度預算,依中央政府特種基金管理準則規定,將空氣污染防制基金由編製單位預算特種基金改制為編製附屬單位預算特種基金\footfullcite{XingZhengYuanZhuJiZongChu}。


隨著世界範圍對於空氣污染、氣候變暖等議題之日益關注,台灣也逐步加强對於空氣污染、溫室氣體之管制。而隨著政策制定、宣導、落實之需求,相關經費來源也造成問題。為加速改善空氣污染問題,空污基金分別於106年4月13日及12月26日提出「空氣污染防制策略」及「空氣污染防制行動方案」,以致空污基金連續3年(107-109年度)入不敷出\footfullcite[「空氣污染防制策略」及「空氣污染防制行動方案」包括輔導老舊車輛改善、加強餐飲業油煙管理、加強道路、營建工程及河川揚塵管理,以及提供優惠貸款鼓勵業者汰換高污染老舊大客貨車等。其基金餘額自106年底97.71億元,降至109年底33.08億元,另倘加計110年度預計短絀9.13億元,則該基金110年底預計餘額為23.95億元。見][]{LiFaYuan2021}。
由於基金連年短絀下餘裕甚有限,故111年度爭取公務預算撥補基金25.35億元,以辦理老舊機車淘汰及柴油車多元改善業務\footnote{包含補助老舊機車淘汰經費14億元(預計淘汰70萬輛,每輛補助2千元);及補助大型柴油車汰換經費11.35億萬元(預計汰換4,080輛,每輛補助37萬5千元,差額由空污基金負擔)。同前注。}。
然而,公務預算撥補空污基金造成了爭議。有立委認爲,在污染者付費的原則之下,空污基金之財源應該來自於對於污染源所徵收之空污費,而不應該由公務預算撥補,相當於以全民稅金負擔\footfullcite[立委洪申翰批評,空污基金不足,環保署還撥補25億元的空污基金,變成用全民稅金替製造業大戶負起應負的責任,根本是亂了套。見][]{ZhongYangTongXunShe2021}
。甚至在立法院對於行政院環保署的年度預算決議中,

進而,溫室氣體減量及管理法之修正案於112年1月10日經立法院三讀通過,更名為《氣候變遷因應法》,並於2月15日經總統公布施行。其中第 28 條,授權中央主管機關為達成國家溫室氣體長期減量目標及各期階段管制目標,得分階段對各類排放溫室氣體之排放源徵收碳費。且依據第 32 條,中央主管機關所設立之溫室氣體管理基金\footnote{現行溫室氣體管理基金收支保管及運用辦法(105年01月30日)
【第 2 條】
本基金為預算法第四條第一項第二款所定之特種基金,隸屬於環境保護基金項下,編製附屬單位預算之分預算,以行政院環境保護署(以下簡稱本署)為主管機關。
【第 4 條】
本基金之用途如下:
一、執行溫室氣體減量工作事項。
二、排放源檢查事項。
三、輔導、補助及獎勵排放源辦理溫室氣體自願減量工作事項。
四、資訊平台帳戶建立、拍賣、配售及交易相關行政工作事項。
五、執行溫室氣體減量及管理所需之約聘僱經費。
六、氣候變遷調適之協調、研擬及推動事項。
七、氣候變遷與溫室氣體減量之教育、宣導及獎助事項。
八、氣候變遷及溫室氣體減量之國際事務。
九、其他有關溫室氣體減量及氣候變遷調適研究事項。
},未來將會以上述碳費作爲財源之一\footnote{氣候變遷因應法(112 年 02 月 15 日)第 32 條:
中央主管機關應成立溫室氣體管理基金,基金來源如下:
一、第二十四條與前條之代金及第二十八條之碳費。
二、第二十五條及第三十六條之手續費。
三、第三十五條拍賣或配售之所得。
四、政府循預算程序之撥款。
五、人民、事業或團體之捐贈。
六、其他之收入。}。


\subsection{研究動機}

在這一背景下,本研究的動機是探討台灣空氣污染防制費的現行制度,以及對該制度的財政管理和政策執行是否需要進行進一步的調整和改進,以更好地實現環保和氣候變遷的目標,並確保財政的合理使用。這個問題至關重要,因為它涉及到國民健康、環境質量、政府財政和公平性等重要議題,需要深入研究和討論。


暫且不論學者對於空污費性質之不同見解
及對於廣徵特別公課現象爲害財政體系之擔憂
\footnote{參柯格鐘,特別公課之概念及爭議-以釋字第四二六號解釋所討論之空氣污染防制費為例,月旦法學雜誌,第 163 期 ,2008年11月,頁194-215。},
縱使經過多次修法,已消除了部分爭議,空氣污染防制費相關之法律規範體系依舊存在一些問題。
% 本文所選取之案例是關於固定污染源空氣污染防制費,故下文聚焦於此。

\subsection{研究範圍}



\subsubsection{行政院環保署空氣污染防制基金}

依據空氣污染防制法第17條:
空氣污染防制費除營建工程由直轄市、縣(市)主管機關徵收外,由中央主管機關徵收\footnote{空氣污染防制法第 2 條:本法所稱主管機關:在中央為行政院環境保護署;在直轄市為直轄市政府;在縣(市)為縣(巿)政府。}。中央主管機關由固定污染源所收款項,應以百分之六十比率將其撥交該固定污染源所在直轄市、縣(市)主管機關;由移動污染源所收款項,應以百分之二十比率將其撥交該移動污染源使用者設籍地或油燃料銷售地所在直轄市、縣(市)主管機關。

依據空氣污染防制法第18條,空氣污染防制費專供空氣污染防制之用,各級主管機關得成立基金管理運用,並成立基金管理會監督運作,


本文梳理空氣污染防制基金之法規範體系,聚焦於中央空污基金,探討其法制上之規範合理性與目前所遇爭議與困境。



\section{特種基金之意義及法律依據}

\subsection{空氣污染防制基金在預算法中之定位}


\subsection{特種基金與特別公課}



\section{特種基金與}


\pagebreak

\subsection{空氣污染防制基金之用途}


\begin{table}[htbp]
  \centering
  \caption{空氣污染防制基金收支保管及運用辦法關於基金用途之條文 修法前后对照}
  \hspace{12pt}% 增加标题与表格之间的距离
  \begin{tabular}{|p{7.5cm}|p{7.5cm}|}
  \hline
  % \multicolumn{1}{|c|}{條文} & 
  \multicolumn{1}{|c|}{第 5 條(90 年 07 月 24 日)} & \multicolumn{1}{c|}{第 4 條(110 年 08 月 12 日)} \\
  \hline
  %  &
  本基金之用途如下:
    \begin{enumerate}[label=\zhnum*、,topsep=0.5em, partopsep=0pt, itemsep=0pt, parsep=0pt,leftmargin=3em]
  \item  關於主管機關執行空氣污染防制工作事項。
  \item  關於空氣污染源查緝及執行成效之稽核事項。
  \item  關於補助及獎勵各類污染源辦理空氣污染改善工作事項。
  \item  關於委託或補助檢驗測定機構辦理汽車排放空氣污染物檢驗事項。
  \item  關於委託或補助專業機構辦理固定污染源之檢測、輔導及評鑑事項。
  \item  關於空氣污染防制技術之研發及策略之研訂事項。
  \item  關於涉及空氣污染之國際環境保護工作事項。
  \item  關於空氣品質監測及執行成效之稽核事項。
  \item  關於徵收空氣污染防制費之相關費用事項。
  \item  關於營建工程棄土場之設置事項。
  \item  執行空氣污染防制相關工作所需人力之聘僱事項。
  \item  其他有關空氣污染防制工作事項。
  \end{enumerate}  
  & 
  本基金之用途如下:
  \begin{enumerate}[label=\zhnum*、,topsep=0.5em, partopsep=0pt, itemsep=0pt, parsep=0pt,leftmargin=3em]
  \item  關於主管機關執行空氣污染防制工作事項。
  \item  關於空氣污染源查緝及執行成效之稽核事項。
  \item  關於補助及獎勵各類污染源辦理空氣污染改善工作事項。
  \item  關於委託或補助檢驗測定機構辦理汽車排放空氣污染物檢驗事項。
  \item  關於委託或補助專業機構辦理固定污染源之檢測、輔導及評鑑事項。
  \item  關於空氣污染防制技術之研發及策略之研訂事項。
  \item  關於涉及空氣污染之國際環保工作事項。
  \item  關於空氣品質監測及執行成效之稽核事項。
  \item  關於徵收空氣污染防制費之相關費用事項。
  \item  執行空氣污染防制相關工作所需人力之聘僱事項。
  \item  關於空氣污染之健康風險評估及管理相關事項。
  \item  \textbf{關於潔淨能源使用推廣及研發之獎勵事項。}
  \item  關於空氣污染檢舉獎金事項。
  \item  關於辦理各項空氣污染改善之貸款信用保證事項。
  \item  其他有關空氣污染防制工作事項。
  \end{enumerate} \\
  \hline
  \end{tabular}
  \end{table}


\pagebreak


\section{特種基金之實例}


中央畜產會\footfullcite{CaiMaoYin1998}

\subsection{特種基金}

以空氣污染防制法之空氣污染防制基金\footfullcite[][]{KeGeZhong2008}


空氣污染防制法\footfullcite[][]{KeGeZhong2008}


第 18 條\textcite{KeGeZhong2008}

% 空氣污染防制基金收支保管及運用辦法
% 第1條
% 為防制空氣污染,維護生活環境及國民健康,以提高生活品質,特依空氣污染防制法(以下簡稱本法)第十八條第二項規定,設置空氣污染防制基金(以下簡稱本基金),並依預算法第二十一條規定,訂定本辦法。


% 預算法 第 21 條

% 政府設立之特種基金,除其預算編製程序依本法規定辦理外,其收支保管辦法,由行政院定之,並送立法院。

% \section{特種基金在預算法上之地位}

% \section{特種基金在稅法上之討論}


\section{空氣污染防制基金作爲特種基金之正當性}


\section{空氣污染防制基金}
\subsection{收入面}

\subsubsection{空氣污染防制費}

首先厘清空污費所應適用之法律原則。

\subsubsubsection{法律保留原則}

釋字第426號解釋,對於空污費之徵收,采取適用相對法律保留原則之見解,并且多數意見認爲,依時空污法第10條之規定
\footnote{空氣污染防制法(中華民國81年01月16日)第10條第1項:「各級主管機關應依污染源排放空氣污染物之種類及排放量,徵收空氣污染防制費用。」,第2項:「前項污染源之類別及收費辦法,由中央主管機關會商有關機關定之。」}
,再參酌法律全部內容,其徵收目的、對象、場所及用途等項,尚難謂有欠具體明確,故未違反法律保留原則。
然而,在釋字第426號解釋的部分不同意見書中,可以看到,戴東雄大法官認爲時空污法第10條之規定尚有不明確之處,如費率之評定及徵收期限,應一併於授權母法中明定為當
\footnote{節錄釋字第426號解釋之戴東雄大法官部分不同意見書:「反之,特別公課之金錢負擔並無憲法之直接明文,咸認其來自憲法第二十三條之法律保留原則之規定。準此以解,特別公課之合憲性較稅捐薄弱,但因其所受立法監督較稅捐不嚴,故為避免行政機關假課徵公課之名,而達增加財政收入之實,並防止財政憲法遭受破壞與架空,公課之徵收仍應有法律保留之正當性,以確保人民之財產權。有鑑於此,公課徵收之目的、用途、對象、費率評定之原則與期限等項,應以法律予以規定。其由法律授權命令訂定者,其授權應符合具體明確始可。多數意見認為空氣污染防制法第十條授權各級主管機關應依污染源排放空氣污染物之種類及排放量徵收空氣污染防制費,從該法整體所表現之關聯性判斷,尚難謂欠具體明確。依本席之見解,即使從整體所表現之關聯性觀察,尚有不明確之處,如費率之評定及徵收期限,應一併於授權母法中明定為當。」}。


% 而參照釋字第593號解釋,對於汽車燃料使用費,

另有釋字第788號解釋,在性質較爲相近的廢棄物清理法回收清除處理費案,針對相對法律保留原則做了更具體的闡述\footnote{節錄釋字第788號解釋之解釋文與理由書:「廢棄物清理法第16條第1項中段所定之回收清除處理費,係國家對人民所課徵之金錢負擔,人民受憲法第15條保障之財產權因此受有限制。其課徵目的、對象、費率、用途,應以法律定之。考量其所追求之政策目標、不同材質廢棄物對環境之影響、回收、清除、處理之技術及成本等各項因素,涉及高度專業性及技術性,立法者就課徵之對象、費率,非不得授予中央主管機關一定之決定空間。故如由法律授權以命令訂定,且其授權符合具體明確之要求者,亦為憲法所許。」}。回收清除處理費同屬於國家為一定政策目標所需,對於有特定關係之人民所課徵之公法上金錢負擔。雖然解釋文仿照釋字第593號解釋(汽車燃料使用費案)之作法,未出現特別公課之用語,但依據定義,回收清除處理費亦屬於特別公課\footnote{請參考許志雄大法官之釋字第788號解釋協同意見書。}。而在788號解釋之中,在相對法律保留原則之下,立法者授予中央主管機關一定決定空間之事項,應僅限於高度專業性、技術性之事項。

而反觀現行條文
\footnote{空氣污染防制法(中華民國107年06月25日)第16條第1項:「各級主管機關得對排放空氣污染物之固定污染源及移動污染源徵收空氣污染防制費,其徵收對象如下:一、固定污染源:依其排放空氣污染物之種類及數量,向污染源之所有人徵收,其所有人非使用人或管理人者,向實際使用人或管理人徵收;其為營建工程者,向營建業主徵收;經中央主管機關指定公告之物質,得依該物質之銷售數量,向銷售者或進口者徵收。二、移動污染源:依其排放空氣污染物之種類及數量,向銷售者或使用者徵收,或依油燃料之種類成分與數量,向銷售者或進口者徵收。」,第2項:「空氣污染防制費徵收方式、計算方式、申報、繳費流程、繳納期限、繳費金額不足之追補繳、收費之污染物排放量計算方法及其他應遵行事項之辦法,由中央主管機關會商有關機關定之。」},歷經數次修法,在第16條第2項,其授權範圍包括了「空氣污染防制費徵收方式、計算方式、申報、繳費流程、繳納期限、繳費金額不足之追補繳、收費之污染物排放量計算方法及其他應遵行事項」,這些事項并非限於涉及高度專業性及技術性者。仔細分辨可以發現,經授權而訂定於收費辦法中之事項,
% 確實包括了本法16條第2項所列者,
例如說徵收期限,性質上屬於公法上請求權時效,攸關人民之基本權利而非屬於高度專業性及技術性之事項,也以收費辦法之形式被規範。針對這部分,本文後續章節將進行更詳細的論述。


總言之,本文認爲,為符合法律保留原則,
空氣污染防制法對於空污費之徵收,僅得就涉及高度專業性及技術性之事項授權予中央主管機關以收費辦法之形式規範,方符合法律保留原則,遵循憲法保障人民基本權利之意旨\footnote{請參考許志雄大法官之釋字第788號解釋協同意見書之末段:「附帶一言,聲請人之一認釋字第 426 號解釋就特別公課之層級化法律保留密度有予以補充解釋之必要,而聲請補充解釋。關於此部分,本號解釋雖敘明不予受理,但對照兩號解釋之解釋文第一段內容可知,其實質意涵已有微妙變化。}。

\subsubsubsection{污染者付費原則}
% (本節略,同上學期報告)
空污費之課徵涉及對於人民財產權等基本權利之限制,
除須受法律保留原則之拘束,亦應符合憲法第7條平等原則及第23條比例原則方屬合憲。
釋字第 426 號解釋確認空污費之課徵是本於污染者付費原則。文本試著將「污染者付費原則」理解爲憲法平等原則在環境公課之具體實現,對標作爲稅法基本原理原則之「量能課稅原則」。

依據污染者付費原則,空污費之計費所應考量者應該是污染者之污染做造成的影響。
鑒於要以污染程度計算出污染者應該負擔之消除污染、復育環境和損害賠償之具體費用是困難複雜的問題,目前所徵收之空污費在具體數額上與各類污染之防治及環境復育費用
% \footnote{環境基本法第28條:環境資源為全體國民世代所有,中央政府應建立環境污染及破壞者付費制度,對污染及破壞者徵收污染防治及環境復育費用,以維護環境之永續利用。}
難以相等同。空污費之課徵僅能在一定程度上作爲經濟誘因的管制手段而促使業者減少排放量
\footnote{戴奧辛空污費遭批太低,環署:減排是目的,
見:\url{https://www.cna.com.tw/news/ahel/201807290090.aspx}.}。

由此,「污染者付費原則」對於空污費之課徵,雖然難以體現在絕對的費用計算(因所徵收之費額不一定相當於污染防治及環境復育等費用),但仍然應該符合平等原則。具體而言,對於不同之污染者,其空污費之計算應該依照法律規定,以污染源、空氣污染之類型、排放量等依據而計算。如此計算得到之公課負擔義務才是在環境公課的概念中,依據事物之本質而對污染者進行合理的差別對待,符合法治國家平等原則之要求。



\subsubsection{環保署撥補}




\subsection{支出面}

\subsubsection{基金用途重合}

空氣污染防制與溫室氣體\footnote{溫室氣體被認爲是廣義上的空氣污染物。}減量之間存在緊密的關係。空氣污染和溫室氣體通常來自相同的源頭,像是油車(移動污染源)、工業製程(固定污染源),很難明確地區分污染源所造成的是空氣污染和溫室氣體中其中何者。也因此,減少這些源頭的污染可以同時減少空氣污染和溫室氣體的排放。例如,化石燃料是主要的溫室氣體排放源,同時也釋放有害的空氣污染物。因此,減少化石燃料的使用可以降低溫室氣體排放,同時減少空氣污染。當政府制定政策和法規,通過提高燃料效率、鼓勵可再生能源的使用、限制排放標準、鼓勵技術創新(例如清潔能源技術與設備的推廣)等方式,對於空污與溫室氣體的減量,效果將會是交互加成的,也就是説,這些手段同時有助於減少空氣污染和溫室氣體排放。因此,本文認爲,今後在區分空污基金和溫室氣體管理基金的用途上,將會有一定的困擾。

具體以上文提到的空污基金所辦理之老舊機車淘汰及柴油車多元改善業務而言,這些輔導政策,由空污基金辦理,有其合理性,因爲老舊油車對於空氣污染來説的確是一類重要的移動污染源,此類業務確實屬於空氣污染防制基金收支保管及運用辦法所定之基金用途。但是同時,此業務亦有助於達成溫室氣體減量之目的。如果是以溫室氣體基金來辦理,亦不違背其基金用途。


\subsubsection{并非符合稅法上對於特別公課概念之界定}

大量的空氣污染防制業務,已經



\section{結論}


\subsection{空污費、碳費應以指定用途稅之性質徵收}
\subsection{强化特種基金之監督}




\pagebreak
\nocite{*}




\section*{參考資料}


% \printbibheading[title={參考資料}]

\printbibliography[type=book, title={專書}]

\printbibliography[type=article, title={期刊文章}]

\printbibliography[type=thesis,title={學位論文}]

\printbibliography[type=立法院,title={立法院報告}]

\printbibliography[keyword=news,title={新聞}]

\printbibliography[nottype=book,nottype=article,nottype=thesis,title={網路資料}]


% \printbibliography[title={其他}, nottype=article]




% \defbibheading{web}{\section*{Web resources}}
% \printbibliography[type = misc,heading=web]



\end{document}